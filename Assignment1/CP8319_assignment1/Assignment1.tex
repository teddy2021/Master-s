\documentclass{article}

\author{Kody Manastyrski}
\date{January 2022}
\title{CP 8319 Assignment 1}

\usepackage{amsmath}
\usepackage{amssymb}
\usepackage{float}

\begin{document}

\maketitle

\section{Question 1}
\subsection{A}
The following matrix shows the values of $r_s$, and is treated as a lookup table.
\[
\begin{bmatrix}
	0 & 0 & 0 & 0 \\
	-5 & -1 & 0 & 1 \\
	0 & 0 & 1 & 5 \\
	0 & 0 & 0 & 0
\end{bmatrix}
\]
Note: Here the $r_g$ and $r_s$ values are included.

Below is the values for each square (i.e. shortened form for the answer), 
followed by the work to obtain them.
\[
\begin{bmatrix}
	6 & 6 & 6 & 6 \\
	-5 & 5 & 6 & 6 \\
	6 & 6 & 6 & 6 \\
	6 & 6 & 6 & 6
\end{bmatrix}
\]

\begin{table}[H]
\begin{tabular}{|l|c|r|}
	\hline
	Cell & Values on Path (Self included) & Total\\
	\hline
	1 & 0 + 0 + 0 + 0 + 1 + 5 & 6\\
	2 & 0 + 0 + 0 + 1 + 5 & 6 \\
	3 & 0 + 0 + 1 + 5 & 6 \\
	4 & 0 + 1 + 5 & 6 \\
	5 & -5 & -5 \\
	6 & -1 + 0 + 1 + 5 & 5 \\
	7 & 0 + 1 + 5 & 6 \\
	8 & 1 + 5 & 6 \\
	9 & 0 + 0 + 1 + 5 & 6 \\
	10 & 0 + 1 + 5 & 6 \\
	11 & 1 + 5 & 6 \\
	12 & 0 + 0 + 0 + 1 + 5 & 6 \\
	13 & 0 + 0 + 0 + 0 + 1 + 5 & 6 \\
	14 & 0 + 0 + 0 + 0 + 0 + 1 + 5 & 6 \\
	15 & 0 + 0 + 0 + 0 + 0 + 0 + 1 + 5 & 6 \\
	16 & 0 + 0 + 0 + 0 + 0 + 0 + 0 + 1 + 5 & 6 \\
	\hline
\end{tabular}
\end{table}


\subsection{B}
\[
\begin{bmatrix}
	18 & 16 & 14 & 12 \\
	-3 & 13 & 12 & 10 \\
	14 & 12 & 10 & 7 \\
	16 & 18 & 20 & 22
\end{bmatrix}
\]

\subsection{C}
\[
	V^{\pi}_{new[i,j]} = (c +\gamma V^{\pi}_{old[i,j]}) + \displaystyle\sum^{|\mathfrak{P}|}_{p \in \mathfrak{P}, x = 0} c + \gamma^x V^{\pi}_{old[p]}
\]
Where $[i,j]$ is the indices of the $V^{\pi}_{old}$ table, and $\mathfrak{P}$ is the ordered sequence of indices for the optimal path from the next step in the path starting at $[i,j]$.
For example, with the above new, starting at $[i=0,j=0]$ we have $\mathfrak{P}=
\{[1,0], [2,0], [3,0],[3,1]\}$ assuming indices start at 0, are in (row, column)
order, and grow moving to the right for rows, and down for columns.
Note that the x goes from 0 to the count of elements in the pathset, while p is
just the indices at each xth step.
\end{document}

\subsection{D}
Supposing that the gridworld now has $r_s = 2 \forall r$ then the world changes
to reward the longest path, since all cells have the same reward value.
On the other hand, if the world merely adds 2 to the existing old rewards, then 
the gridworld disincentives the terminal states, and incentives moving away
from the terminals. 
In the former case, the shaded squares being part of all squares means they are
valued as 2.
In the latter case, the shaded squares have a value of -3 for square 5, and 7 for
square 12.
